\documentclass[12pt]{article}

%
%Margin - 1 inch on all sides
%
\usepackage[letterpaper]{geometry}
\usepackage{times}
\geometry{top=1.0in, bottom=1.0in, left=1.0in, right=1.0in}

%
%Doublespacing
%
\usepackage{setspace}
\doublespacing

%
%Rotating tables (e.g. sideways when too long)
%
\usepackage{rotating}


%
%Fancy-header package to modify header/page numbering (insert last name)
%
\usepackage{fancyhdr}
\pagestyle{fancy}
\lhead{} 
\chead{} 
\rhead{Guo \thepage} 
\lfoot{} 
\cfoot{} 
\rfoot{} 
\renewcommand{\headrulewidth}{0pt} 
\renewcommand{\footrulewidth}{0pt} 
%To make sure we actually have header 0.5in away from top edge
%12pt is one-sixth of an inch. Subtract this from 0.5in to get headsep value
\setlength\headsep{0.333in}

%
%Works cited the environment
%(to start, use \begin{workscited...}, each entry preceded by \bibent)
% - from Ryan Alcock's MLA style file
%
\newcommand{\bibent}{\noindent \hangindent 40pt}
\newenvironment{workscited}{\newpage \begin{center} Works Cited \end{center}}{\newpage }


%
%Begin document
%
\begin{document}
\begin{flushleft}

%%%% First-page name, class, etc
Wenqi Guo\\
Dr. Ethan Guagliardo\\
ENGL 112\\
Feb 18, 2023\\

%%%%Title
\begin{center}
The Root Problem of The Ecological Crisis
\end{center}



%%%%Changes paragraph indentation to 0.5in
\setlength{\parindent}{0.5in}
%%%%Begin the body of the paper here

% Agree 

% human is the real probblem
% xkoukunfaruanxkoukunfadou alexhcuquouculai chouwei  xtiaoshouzhikunaliangcafine chaojikunkunsuankunhuiyi skaole shouruanxkoukunxiangqilaieldanao
% Cap zi ben jia s zhe yi guo cheng xtiao kunxkouzangkunxtiarukukk xkojiaaunn  rende guan dian shi ren de guandian   zhineng neng zibenjiejuejiaosaunnshouzhang leng 
In Moore's article, he disscussed the cause of current ecologtical crisis (EC). He argued that the current EC is created by the the capitalism, not human. He states that the Anthropocene’s view point is that the EC is caused by human as a whole since Industrial Revolution, and human, including capitalism, is not part of the nature. On the other hand, the Capitalocene's argument is that  
Although I agree with the fact that EC aggravated after the present of captalism (Industrial Revolution, per Moore's paper), I do not accept his underriding assumption about the underline cause of the surface phenomenons and problems, i.e. human.

\newpage



\begin{workscited}

\bibent Moore, J. W. The Capitalocene, Part I: On the Nature and Origins of Our Ecological Crisis. The Journal of Peasant Studies 2017, 44 (3), 594–630. https://doi.org/10.1080/03066150.2016.1235036.





\end{workscited}

\end{flushleft}
\end{document}
\}