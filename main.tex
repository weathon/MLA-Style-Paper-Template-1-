\documentclass[12pt]{article}

%
%Margin - 1 inch on all sides
%
\usepackage[letterpaper]{geometry}
\usepackage{times}
\geometry{top=1.0in, bottom=1.0in, left=1.0in, right=1.0in}
\newcommand{\IR}{Industrial Revolution}
\newcommand{\EC}{Ecological Crisis}
%
%Doublespacing
%
\usepackage{setspace}
\doublespacing

%
%Rotating tables (e.g. sideways when too long)
%
\usepackage{rotating}


%
%Fancy-header package to modify header/page numbering (insert last name)
%
\usepackage{fancyhdr}
\pagestyle{fancy}
\lhead{} 
\chead{} 
% \lhead{\LaTeX}
\rhead{Guo \thepage} 
\lfoot{} 
\cfoot{} 
\rfoot{} 
\renewcommand{\headrulewidth}{0pt} 
\renewcommand{\footrulewidth}{0pt} 
%To make sure we actually have header 0.5in away from top edge
%12pt is one-sixth of an inch. Subtract this from 0.5in to get headsep value
\setlength\headsep{0.333in}


%Works cited the environment
%(to start, use \begin{workscited...}, each entry preceded by \bibent)
% - from Ryan Alcock's MLA style file
%
\newcommand{\bibent}{\noindent \hangindent 40pt}
\newenvironment{workscited}{\newpage \begin{center} Works Cited \end{center}}{\newpage }


%
%Begin document
%
\begin{document}
\begin{flushleft}

%%%% First-page name, class, etc
Wenqi Guo\\
Dr. Ethan Guagliardo\\
ENGL 112\\
Feb 19, 2023\\

%%%%Title
\begin{center}
The Root Problem of The Ecological Crisis
\end{center}



%%%%Changes paragraph indentation to 0.5in
\setlength{\parindent}{0.5in}
%%%%Begin the body of the paper here

% Agree 

% human is the real probblem
% xkoukunfaruanxkoukunfadou alexhcuquouculai chouwei  xtiaoshouzhikunaliangcafine chaojikunkunsuankunhuiyi skaole shouruanxkoukunxiangqilaieldanao
% Cap zi ben jia s zhe yi guo cheng xtiao kunxkouzangkunxtiarukukk xkojiaaunn  rende guan dian shi ren de guandian   zhineng neng zibenjiejuejiaosaunnshouzhang leng 
In Moore's article, he discussed the cause of the current ecological crisis. He argued that the current \EC is created by capitalism, not humans. He states that the Anthropocene's viewpoint is that the \EC has been caused by human as a whole since Industrial Revolution, and human, including capitalism, is not part of nature. On the other hand, his argument (i.e. Capitalocene) is that political issues but not technology are the problems, and the fact that capitalism treats nature as cheap resources created the \EC. (Tang) (Moore)
Although I agree with the fact that \EC aggravated after the presence of capitalism and it is contributed by political issues and decisions, I do not accept his underlying assumption about the underline cause of the surface phenomenons and problems, i.e. human and the society of it.


Modern capitalism did enhance \EC, this is very obvious in history. I also agree with his viewpoint that political decisions are involved in the history of development. (Tang) (Moore). However, this is not the only thing that made history, i.e. the cardinality of the vector set involved in the historical decision is larger than what Moore described in his article. There are two decision processes. First, we can imagine the history is in non-complete metric space, which satisfies nonnegativity (two points in history cannot be negatively distanced) and symmetry (the distance in history from point $a$ to $b$ is same as from $b$ to $a$).
The history development process could be seen as the results of virtual forces, which are different components in the decision tree, including the policy (capitalism decisions) and technology (human capability). The direction and magnitude of the process are based on the resultant of the vectors at this moment. The smaller the angle between the policy and technology, the faster the move is; when this angle is small, we can consider the technology contributed to the political decisions, and vice versa: when the angle is large, technology and the political decision limit each other.  At the beginning of \IR, the first situation is the case. However, nowadays, the second situation is more dominant in modern society. A thing to notice is that if we imagine the history space with other dimensions as the welfare in different interest groups of the moment, the capital decision will most likely be the scalar multiplication of the gradient of the welfare dimension with the greatest scalar value of the projection value by the capital interest vector. That is, the capital will choose the best decision for them and only for them in the shortest time frame, which could lead to the local optimum in the metric space with a time dimension, avoiding the global optimum.  
That aligns with Moore's argument. However, he underlooked the force of humans.
The human/technology vector is independent of this, and the limitation effect (the second situation is the composition of forces process) of technology could play a significant role and limit the political decision. 

The second way to model the history process is that politics (capitalism) points the direction, but technology (humans) creates the path. Without the path created by humans, capital could have two solutions: to create the technology (means) needed by the direction or choose another direction, usually the second-best option. From this model, we can observe that the human (technology) is the limiting decision component during history. That is, political systems and decisions created the positive options in the system (what "we" want to do), while humans and technology created the negative options (what "we" cannot do).

% $$F_{history}(cor)=\sqrt{\sum_{i\in F(cor)}i^2}$$
His argument also disacknowledges the process toward \EC already began before the  so-called modern capitalism.
Its limitation of harm is not due to the lack of modern capitalism or modern economic system, but the lack of tools and the population size. 
Like almost all species, \textit{Homo sapiens} is greedy: from bare survival at the early ages to advanced livelihood in modern society. Thus, the interaction, which is mostly negative to the environment, between \textit{Homo sapiens} and nature began in their existence: throughout the history of humans, as a form of exploitation. The forms of exploitation from human varies between types of civilizations. Farming civilization uses the land, which could be deforested, and labour forces from animals. Nomadic civilization is also an obvious "use" of nature. That is, human has been creating a negative impact on nature. Before \IR, however, the equilibrium in many places is not broken yet, and thus the \EC did not appear. In another word, the current \EC is a problem of \textit{Homo sapiens}, and its greedy nature, although the capital amplifies and accelerated this effect. However, it is hard to distinguish between Farming-Livestock Civilization Capitalism and modern capitalism.
\\

% Cap did, but human

In conclusion, I do agree with Moore's argument that the capital created the current \EC. However, I cannot accept the fact that he ignored the importance of technology, or humans with means, in his argument. I used two models to explain the connection between political systems and humans, arguing that a political system can only make decisions when the human virtual force has a small angle with the capital virtual force and only when technologies made the decisions possible. I further argued that his disconnection of human nature and capitalism is not acceptable, as a human is always greedy and this is the basic mechanism of capitalism. That is, although capitalism might seem to be the problem, human nature is what made it possible. I then concluded that human is the root cause of the current \EC and capitalism's immediate cause, as it exposes the weakness of human nature. 
\newpage



\begin{workscited}

\bibent Moore, J. W. The Capitalocene, Part I: On the Nature and Origins of Our Ecological Crisis. The Journal of Peasant Studies 2017, 44 (3), 594–630. doi: 10.1080-03066150.2016.1235036.

\bibent 
 Tang, "Jason W. Moore's Visit to Taiwan: Anthropocene or Capitalocene?," Medium, May 24, 2019. [Online]. Available: https://tgc0927.medium.com/6b9afe8336fd. [Accessed: Feb. 18, 2023]

% \bibent 
% J. Uhlmann, W. Guo "The Application of Metric Search On Dating
% Matching Algorithm." 2023. Student paper.


\end{workscited}

\end{flushleft}
\end{document}
\}